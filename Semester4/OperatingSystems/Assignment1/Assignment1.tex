\documentclass{article}

\usepackage{fancyhdr}
\usepackage{extramarks}
\usepackage{amsmath}
\usepackage{amsthm}
\usepackage{amsfonts}
\usepackage{tikz}
\usepackage[plain]{algorithm}
\usepackage{algpseudocode}
\usepackage{xcolor}
\usepackage{enumitem}
\usepackage{amssymb}
\usepackage{todonotes}
\usepackage{mathtools}
\usepackage{wasysym}
\usepackage{cancel}
\usepackage{phaistos}
\usepackage{listings}
\usepackage[usestackEOL]{stackengine}[2013-10-15]
\def\x{\hspace{6ex}}    %BETWEEN TWO 1-DIGIT NUMBERS
\def\y{\hspace{4.9ex}}  %BETWEEN 1 AND 2 DIGIT NUMBERS
\def\z{\hspace{3.8ex}}    %BETWEEN TWO 2-DIGIT NUMBERS
\stackMath

\usetikzlibrary{automata,positioning}

%
% Basic Document Settings
%

\topmargin=-0.45in
\evensidemargin=0in
\oddsidemargin=0in
\textwidth=6.5in
\textheight=9.0in
\headsep=0.25in

\linespread{1.1}

\pagestyle{fancy}
\lhead{\hmwkAuthorName}
\chead{\hspace{2.5cm} \hmwkClass\ (\hmwkClassInstructor): \hmwkTitle}
\rhead{\firstxmark}
\lfoot{\lastxmark}
\cfoot{\thepage}

\renewcommand\headrulewidth{0.4pt}
\renewcommand\footrulewidth{0.4pt}

\setlength\parindent{0pt}

%
% Create Problem Sections
%

\newcommand{\enterProblemHeader}[1]{
    \nobreak\extramarks{}{Question \arabic{#1} continued on next page\ldots}\nobreak{}
    \nobreak\extramarks{Question \arabic{#1} (continued)}{Question \arabic{#1} continued on next page\ldots}\nobreak{}
}

\newcommand{\exitProblemHeader}[1]{
    \nobreak\extramarks{Question \arabic{#1}}{Question \arabic{#1} continued on next page\ldots}\nobreak{}
    \stepcounter{#1}
    \nobreak\extramarks{Question \arabic{#1}}{}\nobreak{}
}

\newcount\colveccount
\newcommand*\colvec[1]{
        \global\colveccount#1
        \begin{pmatrix}
        \colvecnext
}
\def\colvecnext#1{
        #1
        \global\advance\colveccount-1
        \ifnum\colveccount>0
                \\
                \expandafter\colvecnext
        \else
                \end{pmatrix}
        \fi
}

\setcounter{secnumdepth}{0}
\newcounter{partCounter}
\newcounter{homeworkProblemCounter}
\setcounter{homeworkProblemCounter}{1}
\nobreak\extramarks{Question \arabic{homeworkProblemCounter}}{}\nobreak{}

%
% Homework Problem Environment
%
% This environment takes an optional argument. When given, it will adjust the
% problem counter. This is useful for when the problems given for your
% assignment aren't sequential. See the last 3 problems of this template for an
% example.
%
\newenvironment{homeworkProblem}[1][-1]{
    \ifnum#1>0
        \setcounter{homeworkProblemCounter}{#1}
    \fi
    \section{Question \arabic{homeworkProblemCounter}}
    \setcounter{partCounter}{1}
    \enterProblemHeader{homeworkProblemCounter}
}{
    \exitProblemHeader{homeworkProblemCounter}
}

%
% Homework Details
%   - Title
%   - Due date
%   - Class
%   - Section/Time
%   - Instructor
%   - Author
%

\newcommand{\hmwkTitle}{Assignment 1}
\newcommand{\hmwkDueDate}{7 March, 2019}
\newcommand{\hmwkClass}{Operating Systems}
\newcommand{\hmwkClassTime}{Fall Semester}
\newcommand{\hmwkClassInstructor}{Prof. Fernando Pedone}
\newcommand{\hmwkAuthorName}{Alessandro Romanelli}

%
% Title Page
%

\title{
    \vspace{2in}
    \textmd{\textbf{\hmwkClass:\ \hmwkTitle}}\\
    \normalsize\vspace{0.1in}\small{Due\ on\ \hmwkDueDate\ at 8:30am}\\
    \vspace{0.1in}\large{\textit{\hmwkClassInstructor}}
    \vspace{3in}
}

\author{\hmwkAuthorName}
\date{}

\renewcommand{\part}[1]{\textbf{\large Part \Alph{partCounter}}\stepcounter{partCounter}\\}

%
% Various Helper Commands
%

% Useful for algorithms
\newcommand{\alg}[1]{\textsc{\bfseries \footnotesize #1}}

% For derivatives
\newcommand{\deriv}[1]{\frac{\mathrm{d}}{\mathrm{d}x} (#1)}

% For partial derivatives
\newcommand{\pderiv}[2]{\frac{\partial}{\partial #1} (#2)}

% Integral dx
\newcommand{\dx}{\mathrm{d}x}

% Alias for the Solution section header
\newcommand{\solution}{\textbf{\large Solution}}

% Probability commands: Expectation, Variance, Covariance, Bias
\newcommand{\E}{\mathrm{E}}
\newcommand{\Var}{\mathrm{Var}}
\newcommand{\Cov}{\mathrm{Cov}}
\newcommand{\Bias}{\mathrm{Bias}}

\begin{document}

\maketitle

\pagebreak

\begin{homeworkProblem}
	One of the responsibilities of an operating system is to manage and hide the specific hardware implementation of a machine from the user: the operating system should run seamlessly on different type of machines without the user being able to notice. Another responsibility is that of providing an environment for the user application to run in terms of services and resources. The last responsibility is to provide the user with an interface to interact with the system. 
\end{homeworkProblem}
\begin{homeworkProblem}
	The distinction between kernel mode and user mode functions as a form of protection for the system because it restricts the commands that a user application (possibly malicious) is able to run to those commands which cannot be harmful.
\end{homeworkProblem}
\begin{homeworkProblem}
	Caches are useful because readings from main memory (or even worse secondary storage systems) are generally much slower than reading from a register in the order of the thousand times. A cache addresses this problem by attempting to "remember" the previously loaded data so that it can be accessed far quicker the next time that the same data needs to be read. Caches are also located physically closer to the CPU, shortening the time needed to access it rather than the main memory.
	
	Because caches are expensive to produce, the amount of data that they are able to store is limited and thus the problem is that the machine must decide what to remember of all the data that is read. Moreover, when multiple processors are present (each with its own cache), the hardware must make sure that these caches are coherent and thus store the same values as each other.
	
	Eliminating the device would be a very bad idea and that would probably be apparent the second the machine is turned off. Since the cache memory is volatile, cutting the power will result in the loss of all the data we had previously stored.
\end{homeworkProblem}
\begin{homeworkProblem}
	\begin{enumerate}[label=\alph*)]
		\item The first security problem when having multiple jobs running concurrently on the same machine is that data from a user might be accessed and overwritten by another user. Another problem might be that malicious (or incompetent) users may cause the system to block, denying its resources to other users.
		\item Under the assumption that the operating system of the time-shared machine is efficient and complex enough to guarantee the fair use of its resources and segregation of user's private data from other users, then it would be possible to achieve the same degree of security, ideally speaking.
	\end{enumerate}
\end{homeworkProblem}
\begin{homeworkProblem}
	The purpose of interrupts is often the one of signalling the CPU that a certain action has happened and that control must be handed over to the appropriate interrupt service routine. An interrupt is hardware generated and there is a limited amount of possible interrupts, whereas a trap is a software-generated interrupt, usually caused by an error or by a program that is invoking a routine of the operating service (system call).
\end{homeworkProblem}
\end{document}
