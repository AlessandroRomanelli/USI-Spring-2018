\documentclass{article}

\usepackage{fancyhdr}
\usepackage{extramarks}
\usepackage{amsmath}
\usepackage{amsthm}
\usepackage{amsfonts}
\usepackage{tikz}
\usepackage[plain]{algorithm}
\usepackage{algpseudocode}
\usepackage{xcolor}
\usepackage{enumitem}
\usepackage{amssymb}
\usepackage{todonotes}
\usepackage{mathtools}
\usepackage{wasysym}
\usepackage{cancel}
\usepackage{phaistos}
\usepackage{listings}
\usepackage[usestackEOL]{stackengine}[2013-10-15]
\def\x{\hspace{6ex}}    %BETWEEN TWO 1-DIGIT NUMBERS
\def\y{\hspace{4.9ex}}  %BETWEEN 1 AND 2 DIGIT NUMBERS
\def\z{\hspace{3.8ex}}    %BETWEEN TWO 2-DIGIT NUMBERS
\stackMath

\usetikzlibrary{automata,positioning}

%
% Basic Document Settings
%

\topmargin=-0.45in
\evensidemargin=0in
\oddsidemargin=0in
\textwidth=6.5in
\textheight=9.0in
\headsep=0.25in

\linespread{1.1}

\pagestyle{fancy}
\lhead{\hmwkAuthorName}
\chead{\hspace{2.5cm} \hmwkClass\ (\hmwkClassInstructor): \hmwkTitle}
\rhead{\firstxmark}
\lfoot{\lastxmark}
\cfoot{\thepage}

\renewcommand\headrulewidth{0.4pt}
\renewcommand\footrulewidth{0.4pt}

\setlength\parindent{0pt}

%
% Create Problem Sections
%

\newcommand{\enterProblemHeader}[1]{
    \nobreak\extramarks{}{Problem \arabic{#1} continued on next page\ldots}\nobreak{}
    \nobreak\extramarks{Problem \arabic{#1} (continued)}{Problem \arabic{#1} continued on next page\ldots}\nobreak{}
}

\newcommand{\exitProblemHeader}[1]{
    \nobreak\extramarks{Problem \arabic{#1}}{Problem \arabic{#1} continued on next page\ldots}\nobreak{}
    \stepcounter{#1}
    \nobreak\extramarks{Problem \arabic{#1}}{}\nobreak{}
}

\newcount\colveccount
\newcommand*\colvec[1]{
        \global\colveccount#1
        \begin{pmatrix}
        \colvecnext
}
\def\colvecnext#1{
        #1
        \global\advance\colveccount-1
        \ifnum\colveccount>0
                \\
                \expandafter\colvecnext
        \else
                \end{pmatrix}
        \fi
}

\setcounter{secnumdepth}{0}
\newcounter{partCounter}
\newcounter{homeworkProblemCounter}
\setcounter{homeworkProblemCounter}{1}
\nobreak\extramarks{Problem \arabic{homeworkProblemCounter}}{}\nobreak{}

%
% Homework Problem Environment
%
% This environment takes an optional argument. When given, it will adjust the
% problem counter. This is useful for when the problems given for your
% assignment aren't sequential. See the last 3 problems of this template for an
% example.
%
\newenvironment{homeworkProblem}[1][-1]{
    \ifnum#1>0
        \setcounter{homeworkProblemCounter}{#1}
    \fi
    \section{Problem \arabic{homeworkProblemCounter}}
    \setcounter{partCounter}{1}
    \enterProblemHeader{homeworkProblemCounter}
}{
    \exitProblemHeader{homeworkProblemCounter}
}

%
% Homework Details
%   - Title
%   - Due date
%   - Class
%   - Section/Time
%   - Instructor
%   - Author
%

\newcommand{\hmwkTitle}{Assignment 7}
\newcommand{\hmwkDueDate}{20 December, 2018}
\newcommand{\hmwkClass}{Probability \& Statistics}
\newcommand{\hmwkClassTime}{Fall Semester}
\newcommand{\hmwkClassInstructor}{Prof. D. Eynard}
\newcommand{\hmwkAuthorName}{\textbf{A. Romanelli} / \textbf{A. Vicini}}

%
% Title Page
%

\title{
    \vspace{2in}
    \textmd{\textbf{\hmwkClass:\ \hmwkTitle}}\\
    \normalsize\vspace{0.1in}\small{Due\ on\ \hmwkDueDate\ at 8:30am}\\
    \vspace{0.1in}\large{\textit{\hmwkClassInstructor}}
    \vspace{3in}
}

\author{\hmwkAuthorName}
\date{}

\renewcommand{\part}[1]{\textbf{\large Part \Alph{partCounter}}\stepcounter{partCounter}\\}

%
% Various Helper Commands
%

% Useful for algorithms
\newcommand{\alg}[1]{\textsc{\bfseries \footnotesize #1}}

% For derivatives
\newcommand{\deriv}[1]{\frac{\mathrm{d}}{\mathrm{d}x} (#1)}

% For partial derivatives
\newcommand{\pderiv}[2]{\frac{\partial}{\partial #1} (#2)}

% Integral dx
\newcommand{\dx}{\mathrm{d}x}

% Alias for the Solution section header
\newcommand{\solution}{\textbf{\large Solution}}

% Probability commands: Expectation, Variance, Covariance, Bias
\newcommand{\E}{\mathrm{E}}
\newcommand{\Var}{\mathrm{Var}}
\newcommand{\Cov}{\mathrm{Cov}}
\newcommand{\Bias}{\mathrm{Bias}}

\begin{document}

\maketitle

\pagebreak

\begin{homeworkProblem}
	\begin{enumerate}
		\item We know from the nature of politics that the distribution that we are dealing with will be a binomial one. As seen in class we can get an estimation of $\hat p$ by computing:
		$$\hat p = \left(\sum_{i=1}^nx_i\right)\cdot \frac1n$$
		Which yields the following proportion for candidate A:
		$$\frac{185}{351} \approx 0.527 = 52.7\% \text{ of the voters}$$
		Against candidate B:
		$$\frac{351-185}{351} \approx 0.473 = 47.3\% \text{ of the voters}$$
		We can see that candidate A will be slightly favoured with a 2.7\% margin of vantage.
		\item We want to find a 95\% confidence interval for the $\hat p$ that we just computed:
		$$\text{se}=\sqrt{\hat p (1- \hat p)/n} = 0.027$$
		We can thus attempt to find an interval so that our confidence results:
		$$P(\hat p - \delta < \hat P < \hat p + \delta) = 0.95$$
		The estimation of the proportion can be treated as a random variable binomially distributed, since each sample would either satisfy our confidence interval or not. We can then treat this distribution as a normal one according to the \textbf{Central Limit Theorem}:
		\begin{center}
			$$\begin{array}{rcl}
				P(\hat p - \delta < \hat P < \hat p + \delta) & = & P\left(-\frac{\delta}{\text{se}} < \frac{\hat P - \hat p}{\text{se}} < \frac{\delta}{\text{se}}\right) \\
				 & = & 2\cdot\Phi\left(\frac{\delta}{\text{se}}\right) -1 \\
			\end{array}$$
		\end{center}
		We now need to find $2\Phi\left(\frac{\delta}{\text{se}}\right) -1 = 0.95$, thus:
		$$\Phi\left(\frac{\delta}{\text{se}}\right) = \frac{1+0.95}2$$
		$$\frac{\delta}{\text{se}} = \Phi^{-1}\left(\frac{1+0.95}{2}\right)$$ 
		$$\delta = 1.96 \cdot \text{se} = 0.053$$
		We have now found the bounds of our interval to be: $[\hat p-\delta, \hat p + \delta] = [0.474, 0.58]$ \\
		Since our $\hat p$ lies within this interval we can be pretty confident that the estimated proportion is favourable to A.
		\item We want to have a margin of error $\delta = 0.02$, thus we'd need to increase our sample population $n$ such that:
		$$\delta = 1.96 \cdot \text{se} = 0.02$$
		$$\text{se} = \frac{0.02}{1.96}$$
		$$\sqrt{\hat p(1- \hat p)/n} = 0.0102$$
		$$\frac{0.249271}{0.00010404} = n$$
		$$n = 2396$$
	\end{enumerate}
\end{homeworkProblem}
\newpage
\begin{homeworkProblem}
	\begin{enumerate}
		\item Given the following data: (13.0, 18.5, 16.4, 14.8, 20.8, 19.3, 18.8, 23.1, 25.1, 16.8, 20.4, 17.4, 16.0, 21.7, 15.2, 21.3, 19.4, 17.3, 23.2, 24.9, 15.2, 19.9, 19.1, 18.1, 25.2, 23.1, 15.3, 19.4, 21.5, 16.8, 15.6, 17.6) we computed the sample mean $\overline x$ and the standard deviation $\sigma$ with the following code:\\
			\begin{lstlisting}
samples <- c(13.0, 18.5, ..., 15.6, 17.6)
mu <- mean(samples)
print(mu)
sigma <- sqrt(var(samples))
print(sigma)
			\end{lstlisting}
		Which gives us $\overline x = 19.068, \sigma = 3.255$
		\item Under the assumption of an underlaying normal distribution and variance $\sigma^2 = 3.12$, we can attempt to find a 95\% confidence interval in the following way:
		$$\left[\overline x - \delta,\overline x + \delta\right]$$
		$$\delta = z^*\frac{\sigma}{\sqrt n}$$
		Where the z value for our 95\% confidence interval will be $z^* = 1.96$:
		$$\delta_{0.95} = 1.96 \cdot \frac{\sqrt{3.12}}{\sqrt{32}} = 1.96 \cdot \frac{1.766}{5.657} = 0.612$$
		Thus we get the resulting :
		$$\text{CI}_{0.95} = [\overline x - \delta_{0.95}, \overline x + \delta_{0.95}] = [19.068 - 0.612, 19.068+0.612] = [18.46, 19.68]$$
		\item In case of a 99\% confidence interval our $z^*$ would be 2.58 instead, which would result in the following margin of error:
		$$\delta_{0.99} = 2.58 \cdot \frac{\sqrt{3.12}}{\sqrt{32}} = 0.806$$
		And it's clearly visible that $\delta_{0.95} < \delta_{0.99}$ and thus we'd have a larger confidence interval than before. This comes from the fact that a wider interval grants us more confidence that our sample lies within the interval than not. \\ \\
		Increasing the sample size $n$ will reduce the margin of error and also tighten the interval.
	\end{enumerate}
\end{homeworkProblem}
\newpage
\begin{homeworkProblem}
\begin{enumerate}[label=\alph*)]
	\item 	Under the assumption of a biased coin with $\hat p = 0.6$, we can compute the probability as the one of a binomial distribution:
	$$P(S_n = n\cdot \hat p) = {n \choose n \cdot \hat p}\hat p^{n\hat p}(1-\hat p)^{n(1-\hat p)}$$
	$$P\left(S_n = \frac35n\right) = {n \choose \frac35n}\left(\frac35\right)^{\frac35n}\left(\frac25\right)^{\frac25n}$$
	For $n=10$:
	$$P(S_{10} = 6) = {10 \choose 6}\left(\frac35\right)^{6}\left(\frac25\right)^{4} \approx 210 \cdot 0.0466 \cdot 0.0256 = 0.25$$
	For $n=100$:
	$$P(S_{100} = 60) = {100 \choose 60}\left(\frac35\right)^{60}\left(\frac25\right)^{40} \approx 0.081$$
	For $n=1000$:
	$$P(S_{1000} = 600) = {1000 \choose 600} \left(\frac35\right)^{600}\left(\frac25\right)^{400} \approx 0.026$$
\end{enumerate}

\end{homeworkProblem}
\newpage
\begin{homeworkProblem}

\end{homeworkProblem}
\newpage
\begin{homeworkProblem}
	To accept or reject H0 we need to compute the z-value:
	$$z = \frac{x -\mu}{\sigma/\sqrt n} \Rightarrow z = \frac{113.5-110}{10/4} = 1.4$$
	If the resulting $z$ is greater than what we just computed, then we'll accept H0, else we reject it. 
	\\\\
	\begin{enumerate}[label=\alph*)]
		\item 	A 5\% significance level corresponds to $\alpha = 0.05$: $z_{5\%} = \Phi^{-1}(0.95) \overset{?}{>} 1.4$
	$$z_{5\%} = 1.65 > 1.4\text{, thus we accept H0}$$
		\item Given $\alpha = 0.10$: $z_{10\%} = \Phi^{-1}(0.90)\overset{?}{>} 1.4$
		$$z_{10\%} = 1.28 < 1.4\text{, thus we reject H0 and accept H1 instead}$$
		\item The p-value corresponds to the right-tail-end probability of the null hypothesis H0:
		$$p = 1 - \Phi(z) = 1 - \Phi(1.4) \approx 1 - 0.919 = 0.081 $$
	\end{enumerate}
\end{homeworkProblem}
\newpage
\begin{homeworkProblem}
	
\end{homeworkProblem}
\end{document}
