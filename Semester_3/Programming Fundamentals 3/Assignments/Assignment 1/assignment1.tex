\documentclass{article}

\usepackage{fancyhdr}
\usepackage{extramarks}
\usepackage{amsmath}
\usepackage{amsthm}
\usepackage{amsfonts}
\usepackage{tikz}
\usepackage[plain]{algorithm}
\usepackage{algpseudocode}
\usepackage{xcolor}
\usepackage{enumitem}
\usepackage{amssymb}
\usepackage{todonotes}
\usepackage{mathtools}
\usepackage{wasysym}
\usepackage{cancel}
\usepackage{listings}

\usetikzlibrary{automata,positioning}

%
% Basic Document Settings
%

\definecolor{mygreen}{rgb}{0,0.6,0}
\definecolor{mygray}{rgb}{0.5,0.5,0.5}
\definecolor{mymauve}{rgb}{0.58,0,0.82}

\lstset{
  backgroundcolor=\color{white},   % choose the background color; you must add \usepackage{color} or \usepackage{xcolor}; should come as last argument
  basicstyle=\footnotesize,        % the size of the fonts that are used for the code
  breakatwhitespace=false,         % sets if automatic breaks should only happen at whitespace
  breaklines=true,                 % sets automatic line breaking
  captionpos=b,                    % sets the caption-position to bottom
  commentstyle=\color{mygreen},    % comment style
  deletekeywords={...},            % if you want to delete keywords from the given language
  escapeinside={\%*}{*)},          % if you want to add LaTeX within your code
  extendedchars=true,              % lets you use non-ASCII characters; for 8-bits encodings only, does not work with UTF-8
  frame=single,	                   % adds a frame around the code
  keepspaces=true,                 % keeps spaces in text, useful for keeping indentation of code (possibly needs columns=flexible)
  keywordstyle=\color{orange!90!black},       % keyword style
  language=Java,                 % the language of the code
  morekeywords={*,...},            % if you want to add more keywords to the set
  numbers=left,                    % where to put the line-numbers; possible values are (none, left, right)
  numbersep=5pt,                   % how far the line-numbers are from the code
  numberstyle=\tiny\color{mygray}, % the style that is used for the line-numbers
  rulecolor=\color{black},         % if not set, the frame-color may be changed on line-breaks within not-black text (e.g. comments (green here))
  showspaces=false,                % show spaces everywhere adding particular underscores; it overrides 'showstringspaces'
  showstringspaces=false,          % underline spaces within strings only
  showtabs=false,                  % show tabs within strings adding particular underscores
  stepnumber=1,                    % the step between two line-numbers. If it's 1, each line will be numbered
  stringstyle=\color{yellow!60!orange!80!black},     % string literal style
  tabsize=2,	                   % sets default tabsize to 2 spaces
  title=\lstname,                   % show the filename of files included with \lstinputlisting; also try caption instead of title
  keywords=[3]{self},
  keywordstyle=[3]{\color{orange!50!gray}},
  keywords=[4]{__init__, __lt__},
  keywordstyle=[4]{\color{orange!50!gray}},
  moredelim=**[is][\color{orange!90!black}]{@}{@}
}


\topmargin=-0.45in
\evensidemargin=0in
\oddsidemargin=0in
\textwidth=6.5in
\textheight=9.0in
\headsep=0.25in

\linespread{1.1}

\pagestyle{fancy}
\lhead{\hmwkAuthorName}
\chead{\hspace{2.5cm} \hmwkClass\ (\hmwkClassInstructor): \hmwkTitle}
\rhead{\firstxmark}
\lfoot{\lastxmark}
\cfoot{\thepage}

\renewcommand\headrulewidth{0.4pt}
\renewcommand\footrulewidth{0.4pt}

\setlength\parindent{0pt}

%
% Create Problem Sections
%

\newcommand{\enterProblemHeader}[1]{
    \nobreak\extramarks{}{Problem \arabic{#1} continued on next page\ldots}\nobreak{}
    \nobreak\extramarks{Problem \arabic{#1} (continued)}{Problem \arabic{#1} continued on next page\ldots}\nobreak{}
}

\newcommand{\exitProblemHeader}[1]{
    \nobreak\extramarks{Problem \arabic{#1} (continued)}{Problem \arabic{#1} continued on next page\ldots}\nobreak{}
    \stepcounter{#1}
    \nobreak\extramarks{Problem \arabic{#1}}{}\nobreak{}
}

\newcount\colveccount
\newcommand*\colvec[1]{
        \global\colveccount#1
        \begin{pmatrix}
        \colvecnext
}
\def\colvecnext#1{
        #1
        \global\advance\colveccount-1
        \ifnum\colveccount>0
                \\
                \expandafter\colvecnext
        \else
                \end{pmatrix}
        \fi
}

\setcounter{secnumdepth}{0}
\newcounter{partCounter}
\newcounter{homeworkProblemCounter}
\setcounter{homeworkProblemCounter}{1}
\nobreak\extramarks{Problem \arabic{homeworkProblemCounter}}{}\nobreak{}

%
% Homework Problem Environment
%
% This environment takes an optional argument. When given, it will adjust the
% problem counter. This is useful for when the problems given for your
% assignment aren't sequential. See the last 3 problems of this template for an
% example.
%
\newenvironment{homeworkProblem}[1][-1]{
    \ifnum#1>0
        \setcounter{homeworkProblemCounter}{#1}
    \fi
    \section{Problem \arabic{homeworkProblemCounter}}
    \setcounter{partCounter}{1}
    \enterProblemHeader{homeworkProblemCounter}
}{
    \exitProblemHeader{homeworkProblemCounter}
}

%
% Homework Details
%   - Title
%   - Due date
%   - Class
%   - Section/Time
%   - Instructor
%   - Author
%

\newcommand{\hmwkTitle}{Assignment 1}
\newcommand{\hmwkDueDate}{October 8, 2018}
\newcommand{\hmwkClass}{PF3}
\newcommand{\hmwkClassTime}{Spring Semester}
\newcommand{\hmwkClassInstructor}{Prof. Walter Binder }
\newcommand{\hmwkAuthorName}{\textbf{A. Romanelli}}

%
% Title Page
%

\title{
    \vspace{2in}
    \textmd{\textbf{\hmwkClass:\ \hmwkTitle}}\\
    \normalsize\vspace{0.1in}\small{Due\ on\ \hmwkDueDate\ at 8:30am}\\
    \vspace{0.1in}\large{\textit{\hmwkClassInstructor}}
    \vspace{3in}
}

\author{\hmwkAuthorName}
\date{}

\renewcommand{\part}[1]{\textbf{\large Part \Alph{partCounter}}\stepcounter{partCounter}\\}

%
% Various Helper Commands
%

% Useful for algorithms
\newcommand{\alg}[1]{\textsc{\bfseries \footnotesize #1}}

% For derivatives
\newcommand{\deriv}[1]{\frac{\mathrm{d}}{\mathrm{d}x} (#1)}

% For partial derivatives
\newcommand{\pderiv}[2]{\frac{\partial}{\partial #1} (#2)}

% Integral dx
\newcommand{\dx}{\mathrm{d}x}

% Alias for the Solution section header
\newcommand{\solution}{\textbf{\large Solution}}

% Probability commands: Expectation, Variance, Covariance, Bias
\newcommand{\E}{\mathrm{E}}
\newcommand{\Var}{\mathrm{Var}}
\newcommand{\Cov}{\mathrm{Cov}}
\newcommand{\Bias}{\mathrm{Bias}}

\begin{document}

\maketitle

\pagebreak

\begin{homeworkProblem}
\begin{enumerate}[label=\arabic*]
	\item 
	\begin{enumerate}[label=.\arabic*]
		\item \texttt{"This is printed in class B"}, it comes from the method print defined in B, which overloads the original method defined in A, but since the type of the argument B, thus the latter is called.
		\item An abstract class cannot be instantiated: \verb|new A()| is an invalid statement.
		\item Argument mismatch: a method that takes a String as type of the argument cannot be found and thus an exception is thrown.
		\item \texttt{"This is printed in class A"}, it comes from the method print defined in A, after an object of class B is casted into an object of class A. This means that the object of class B on which the method is called as both methods for an argument of type A and an argument of type B, since the argument is of type A, the method inherited from class A is used.
		\item \texttt{"This is printed in class A"} is the result. The object a is instantiated from the class B and that's its dynamic type, but it's declared type is A and thus in the method call, the argument type is going to be A and thus the method defined in A will be called.
		\item The method print is not defined within the class Object and would thus throw an exception.
		\item Abstract class A cannot be instantiated.
		\item \texttt{"This is printed in class A"}. This happens because we instantiated an object of class B but cast it into an object of type A, thus not having the method defined in the B class. When we call the method print passing an object of class B, the method A is used anyway because it's the only one available and accepts all subtypes of class A.
		\item Argument mismatch, a method that takes a String as argument cannot be found and thus raises an exception.
		\item Abstract class A cannot be instantiated.
	\end{enumerate}
	\item
	\begin{enumerate}[label=.\arabic*]
		\item Overload, the two methods coexist and both can be used depending on the type of the provided argument.
		\item Override, the two methods share the same argument type, hence when the method print is defined again in C (subclass of A) taking the same argument type, it overrides the previously defined method with the new one.
		\item None, because B and D are not related besides having a common parent of A and thus their methods do not have any interaction.
		\item Override, the method defined in class D was already defined in class C from which it is inheriting and having the same argument type, the latter will override the former.
		\item None, since the method defined in C is private, D does not inherit it and when it attempts to define the same method in class D, it does not override nor overload anything.
	\end{enumerate}
	\item
		\begin{enumerate}[label=.\arabic*]
		\item The declared type is Object and the possible dynamic types of $\circ$ are: Object/A/B/C/D;
		\item The declared type is D and $\circ$ accepts only D as dynamic type;
		\item Declared type C and $\circ$ accepts C and D as dynamic types;
		\item Declared type A and accepts all its subtypes as dynamic types: A, B, C, D;
		\item Declared type B, accepting as dynamic types: B.
	\end{enumerate}
\end{enumerate}

\end{homeworkProblem}
\begin{homeworkProblem}
	\begin{enumerate}[label=\arabic*]
		\item \texttt{IntSet2} violates the SP because it may throw \texttt{DuplicateNotAllowedException}, which the superclass method does not declare and thus this will cause an error at compile time.
		\item No, it does not violate the SP, \texttt{int} is converted into an Integer through auto-boxing and \texttt{IntSet2.insert} overloads \verb|IntList.insert| without changing return type, parameters types or thrown exceptions.
	\end{enumerate}
\end{homeworkProblem}
\newpage
\begin{homeworkProblem}
\begin{lstlisting}
import java.math.BigDecimal;

class BankAccount {

    private BigDecimal amount;

    public BankAccount(BigDecimal amount) {
        this.amount = amount;
    }

    public void transferTo(BankAccount target, BigDecimal amount) {
        // Local variables for Postconditions check
        BigDecimal initialAmount = this.amount;
        BigDecimal initialTargetAmount = target.amount;
        // Invariants (2)
        assert target.amount.compareTo(BigDecimal.ZERO) != -1: "The target BankAccount must have a positive amount";
        assert this.amount.compareTo(BigDecimal.ZERO) != -1: "The current BankAccount must have a positive amount";
        // Preconditions (3)
        assert ((amount.compareTo(BigDecimal.ZERO)) != -1) && (amount.scale() == 2): "Amount to transfer must be positive and have two digits";
        assert this != target: "Transfer target cannot be the same as the origin";
        assert this.amount.compareTo(amount) != -1: "Cannot transfer more money that the BankAccount has";
        // Code
        this.amount = this.amount.subtract(amount);
        target.amount = target.amount.add(amount);
        // Invariants (2)
        assert target.amount.compareTo(BigDecimal.ZERO) != -1: "The target BankAccount must have a positive amount";
        assert this.amount.compareTo(BigDecimal.ZERO) != -1: "The current BankAccount must have a positive amount";
        // Postconditions (2)
        assert this.amount.add(amount).compareTo(initialAmount) == 0: "The money we transfered + the money we have must be equal to the money we originally had";
        assert target.amount.subtract(amount).compareTo(initialTargetAmount) == 0: "The money the target has - the money we transfered must be equal to the money it had";
    }

    public String toString() {
        return amount.toString();
    }
}
\end{lstlisting}
\end{homeworkProblem}
\newpage
\begin{homeworkProblem}
\begin{lstlisting}
class A {
    public String show(Object obj) {
        return ""+this.getClass().getSimpleName()+".show("+obj.getClass().getSimpleName()+")";
    }
}

class B extends A {}

class C extends B {}

class D extends C {}

public class Main {
    public static void main(String[] args) {
        A a1 = new A();
        A a2 = new B();
        B b = new B();
        C c = new C();
        D d = new D();
        System.out.println(a1.show(b)); // 1
        System.out.println(a1.show(c)); // 2
        System.out.println(a1.show(d)); // 3
        System.out.println(a2.show(b)); // 4
        System.out.println(((B) a2).show(b)); // 5
        System.out.println(a2.show(c)); // 6
        System.out.println(a2.show(d)); // 7
        System.out.println(b.show(b)); // 8
        System.out.println(b.show(c)); // 9
        System.out.println(b.show(d)); // 10
    }
}
\end{lstlisting}
This implementation of the A class provides the exact same functionality of initially proposed design.
\end{homeworkProblem}

\end{document}
